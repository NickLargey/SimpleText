% This is samplepaper.tex, a sample chapter demonstrating the
% LLNCS macro package for Springer Computer Science proceedings;
% Version 2.20 of 2017/10/04
%
\documentclass[runningheads]{llncs}
%
\usepackage{graphicx}
\usepackage{tabularx}
% Used for displaying a sample figure. If possible, figure files should
% be included in EPS format.
%
% If you use the hyperref package, please uncomment the following line
% to display URLs in blue roman font according to Springer's eBook style:
% \renewcommand\UrlFont{\color{blue}\rmfamily}

\begin{document}
%
\title{SimpleText Task 3 Report}
%
%\titlerunning{Abbreviated paper title}
% If the paper title is too long for the running head, you can set
% an abbreviated paper title here
%
\author{Nicholas Largey\inst{1}\orcidID{0525664}}
%
\authorrunning{N. Largey}
% First names are abbreviated in the running head.
% If there are more than two authors, 'et al.' is used.
%
\institute{University of Southern Maine, Portland, ME, 04101, USA
\email{nicholas.largey@maine.edu}}

%
\maketitle              % typeset the header of the contribution
%
\begin{abstract}
Phase 1 of our project for COS470/570 is designed so that we the students may get a 
full understanding of the chosen task from this year's CLEF2024 conference. My group and I
have chosen the SimpleText category where I have chosen task 3, Simplifying Scientific Texts, 
as my contribution to the project. Scientific texts can be extremely difficult for the average
person to understand, so this task aims to create or utilize a Deep Learning model to provide 
accurate and digestible simplifications for a person without an extensive background in the
sciences.
\keywords{CLEF2024 \and SimpleText  \and LLMs \and B. Mansouri.}
\end{abstract}
%
%
%
\section{The Task At Hand}
\subsection{A Subsection Sample}
The goal of the SimpleText lab for the 2024 CLEFlabs conference is to make scientific writings 
more accessable to the general public. Task 1 - Retrieving passages to include in a simplified summary - has the goal of filtering out sections, either as little as sentences or as large as entire passages, in order to provide a summary that still gets the overarching purpose, findings and methods in a text across to the reader, without necessarily going into extensive detail. 

Task 2 - Identifying and explaining difficult concepts - is meant to create a model that can extract difficult topics or concepts from a given text and provide an explanation that would be suitable for the "average" English speaking person. 

Task 3 - Simplify Scientific Text - is a mix between the first 2 tasks. Given a scientific text as input, a model needs to be designed or fine-tuned to output a version of that input that is accurate and digestible for a person without and extensive background in the sciences. This will require both approaches taken for task 1 and task 2 since in order to provide a simplified version, the input text will need to be reduced to it's "essence" and which will still contain difficult to understand concepts or wording which will need to be identified and explained.       
%
\section{Previous Approaches}
%

\begin{table}
\caption{Approaches previously taken by participants and other researcher}\label{tab1}
\begin{tabularx}{\textwidth}{|l|X|X|}
\hline
Paper & Strengths & Weaknesses\\
\hline
APEAtSTS[1] & Refined the pipeline and computational resources needed with prompt engineering and were able to attain a 8.43 FKGL & SARI score only reached 47.98 out\\
\hline
ASoSTUPTLLMs[2] & An easy to use and approachable method & Final output wasn't very competitive as no hyper-parameter or fine-tuning was performed.\\
\hline
CSSUTL[3] & An interesting approach to reduce time and resources. & The input text most likely required some of the masked out information for the T5 model to produce the finalized output.\\
\hline
STSaGA[4] & Fairly decent FKGL and SARI results considering the approach & Approach needs more refinement and training \\
\hline
STSUBART[5] & Good approach to data preprocessing saving time and resources & SARI score is given, but FKGL is not so determining how the work compares to others in more difficult. \\
\hline
TSoSTfNER[6] & Very detailed reporting on the approach taken and which models and metrics were used & Results had lots of hallucinations and metrics reported didn't match the required metrics by CLEF \\
\hline
AIAYN[7] & Seminal work on Transformer Models with many excellent visual representations and mathematical explanations & None really.\\
\hline
L2:OFaFTCM[8] & Llama-2 is the SOTA in LLMs and is fully open-source & Paper could have been split into sections that were papers on their own. \\
\hline
QLoRA[9] & Amazing approach to increase efficiency while training LLMs & Can be difficult to implement while fine-tuning models\\
\hline
\end{tabularx}
\end{table}
\subsubsection{Sample Heading (Third Level)} 
GPT4, T5 - Using a Prompt Engineering approach, the authors attempted to refine a the input by first having a GPT4 model simplify the text, then run the simplification through a T5 model for the final output.

SimpleT5, J2, BLOOM - Using preexisting models, the authors attempted to utilize the existing deep learning ecosystem to provide competitive scores

T5, COVID-SciBERT - Using COVID-SciBERT to mask out existing simplified sentences in the input text, the authors would then use T5 to provide simplifications for the rest of the text. 
\section{Planned Approach}
\section{Conclusion}

%
% the environments 'definition', 'lemma', 'proposition', 'corollary',
% 'remark', and 'example' are defined in the LLNCS documentclass as well.
%
\begin{proof}
Proofs, examples, and remarks have the initial word in italics,
while the following text appears in normal font.
\end{proof}
For citations of references, we prefer the use of square brackets
and consecutive numbers. Citations using labels or the author/year
convention are also acceptable. The following bibliography provides
a sample reference list with entries for journal
articles~\cite{ref_article1}, an LNCS chapter~\cite{ref_lncs1}, a
book~\cite{ref_book1}, proceedings without editors~\cite{ref_proc1},
and a homepage~\cite{ref_url1}. Multiple citations are grouped
\cite{ref_article1,ref_lncs1,ref_book1},
\cite{ref_article1,ref_book1,ref_proc1,ref_url1}.
%
% ---- Bibliography ----
%
% BibTeX users should specify bibliography style 'splncs04'.
% References will then be sorted and formatted in the correct style.
%
% \bibliographystyle{splncs04}
% \bibliography{mybibliography}
%
\begin{thebibliography}{9}
\bibitem{ref_article1}
Author, F.: Article title. Journal \textbf{2}(5), 99--110 (2016)
\bibitem{ref_article2}
Author, F.: Article title. Journal \textbf{2}(5), 99--110 (2016)
\bibitem{ref_article3}
Author, F.: Article title. Journal \textbf{2}(5), 99--110 (2016)
\bibitem{ref_article4}
Author, F.: Article title. Journal \textbf{2}(5), 99--110 (2016)
\bibitem{ref_article5}
Author, F.: Article title. Journal \textbf{2}(5), 99--110 (2016)
\bibitem{ref_article6}
Author, F.: Article title. Journal \textbf{2}(5), 99--110 (2016)
\bibitem{ref_article7}
Author, F.: Article title. Journal \textbf{2}(5), 99--110 (2016)
\bibitem{ref_article8}
Author, F.: Article title. Journal \textbf{2}(5), 99--110 (2016)
\bibitem{ref_article9}
Author, F.: Article title. Journal \textbf{2}(5), 99--110 (2016)

\end{thebibliography}
\end{document}
